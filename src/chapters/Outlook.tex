\section{Ausblick}

Ist der Versuch, Shooter-Spieler von einem gewaltfreien Spiel zu �berzeugen, gegl�ckt? Diese Frage ist wohl noch nicht
beantwortet. Der Prototyp ist noch nicht ausgereift genug, um es mit anderen Spielen auf dem Markt aufzunehmen. Auch
haben erst eine kleine Auswahl von Spielern die Gelegenheit gehabt, Cydonia zu testen. Doch die ersten Versuche zeigen,
dass es sich lohnt, den eingeschlagenen Weg weiter zu gehen. 

\subsection{Die n�chsten Schritte}
Damit das Konzept von Cydonia Erfolg haben kann, muss aus dem Prototyp ein fertiges Spiel werden. Viele Details wurden
bei der Entwicklung vernachl�ssigt, einfach weil es zu viel Zeit kostet, sich mit ihnen zu befassen. Der Fokus lag auf
der Untersuchung des gewaltfreien Konzepts und seiner Elemente. Passende Avatarmodelle, Equipment-Modelle, Animationen,
Men�s, indivualisierbare Steuerung, detaillierte Ansicht der verf�gbaren Server, In-Game-Chat, Ping-Anzeige um nur ein
paar der Dinge zu nennen, die noch nicht fertig oder garnicht vorhanden sind. Wie in Abschnitt
\ref{subsubsection:Netcode} bereits erw�hnt, ist der Netzcode nur grundlegend implementiert und m�sste starkt optimiert
werden, vor allem wenn Cydonia auch �ber das Internet gespielt werden soll.

