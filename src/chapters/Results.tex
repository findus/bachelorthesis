\section{Ergebnisse}

Im Rahmen der Tests wurden viele anf�ngliche Probleme und Schwachstellen aber auch positive und erhaltenswerte Elemente
identifiziert und so das Konzept stetig verbessert.

\subsection{Ersatz f�r Schie�en}
In einem Prototypen des Spiels wurden mehrere m�gliche Spielelemente als Ersatz f�r das Schie�en implementiert und Spielern zum Testen gegeben.
Die Tests ergaben f�r die in dem Prototypen implementierten Konzepte f�r die direkte Einwirkung auf den Gegner folgende
Bewertungen:
\begin{itemize} 
\item Beam-Strahl: Die �hnlichkeit mit einer Schusswaffe ist zu hoch, der Unterschied zwischen
Erschie�en und Wegbeamen zu gering.
\item Schub-Strahl: Diese L�sung bietet nicht genug Spielspa�
und funtkioniert nicht in engen R�umen. Au�erdem ist auch hier die �hnlichkeit zu einer Waffe zu hoch.
\item Swapper: Ist vollkommen gewaltfrei und unterst�tzt Kreativit�t und Kooperation.
\end{itemize} 
Die beste Substitution ist ein "`Swapper"' getauftes Equipment, womit der Spieler zwei beliebige Dinge (Gegenst�nde, Mitspieler,
Gegner) markieren kann, die daraufhin den Platz tauschen. Danach sind beide Marker zur�ckgesetzt, m�ssen also neu
platziert werden. Diese Technik stellt nicht nur eine gewaltfreie M�glichkeit dar, den Gegner von seinem Weg
abzubringen, es schafft auch gro�en taktischen und kreativen Freiraum. Die Spieler k�nnen damit die Spielwelt zu ihrem
Vorteil ver�ndern, kooperativ die besten L�sungen finden, sowie den Gegner in die Irre f�hren. Bei der Benutzung sind
genaues Zielen und gutes Timing n�tig, was die gew�nschte haptische �hnlichkeit zu den Waffen des gemeinen Shooters
ergibt. Au�erdem ist es erforderlich, taktisch und vorausplanend mit den beiden verf�gbaren Markern umzugehen, worin
sich eine Analogie zum Haushalten mit Munition im Shooter finden l�sst. Es spricht aber auch nichts dagegen, dem Spieler
mehrere unterschiedliche Equipments anzubieten, um so mehr Abwechslung zu schaffen. Auch andere Spiele bieten
unterschiedliche Waffen an, aus denen der Spieler w�hlen kann. 