\section{Multiplayer-Taktik-Shooter}


\subsection{Definition}
Heutige Computerspiele werden h�ufig einem der folgenden Genres zugesprochen: Shooter, Strategie, Rollenspiele. Diese
Liste ist nat�rlich weder vollst�ndig, noch ist die Zuordnung stets eindeutig. Alle Genres haben diverse Subgenres, es
gibt Spiele, die Elemente aus mehreren Genres vereinen, sog. Hybride. Au�erdem kann in fast jedem Genre noch nach
Singleplayer, Multiplayer oder Massive-Multiplayer unterschieden werden.\\

Um f�r die vorliegende Arbeit Unklarheiten und Missverst�ndnisse so gut es get zu vermeiden, soll im folgenden das
Subgenre der Multiplayer-Taktik-Shooter definiert und zu anderen Genres abgegrenzt werden. Die Definition l�sst sich
leicht anhand der einzelnen Namensbestandteile erl�utern.

\subsubsection{Shooter}
Ein Shooter ist ein Spiel, bei dem der Spieler einen einzelnen Avatar in einer virtuellen Welt steuert. Die
Steuerung erfolgt klassisch �ber Maus und Tastatur, wobei die Maus zum Umsehen und Zielen, die Maustasten zum Feuern der
Waffe und die Tastatur zum Bewegen benutzt wird. Das Spielziel ist meist das T�ten des Gegners mit einer Waffe.
Existiert ein alternatives Spielziel, so ist das Ausschalten des Gegners n�tig oder zumindest zutr�glich f�r das
Erreichen desselben.\\

Eine weitere Unterscheidung l�sst sich anhand der Perspektive feststellen. Von den drei
M�glichkeiten first-person, third-person und top-down ist erstere die bei Shootern am h�ufigsten eingesetzte. Auch
in der vorliegenden Arbeit soll diese verwendet werden.

\subsubsection{Taktik}
W�hrend in einem reinen Ego-Shooter der Spielablauf ausschlie�lich darauf ausgerichtet ist, Gegner zu t�ten und dabei
kein besonderes taktisches Vorgehen verlangt ist, sind in einem Taktik-Shooter unter anderem Fahigkeiten wie geschickte
Positionierung, Timing, Kenntnis der Umgebung, Abstimmung im Team gefordert. Das Schie�en auf den Gegner und die dabei
geforderte Pr�zision sind somit nicht allein spielentscheidend.

\subsubsection{Multiplayer}
Als Multiplayer-Spiel bezeichnet man ein Computerspiel, das �ber Netzwerk oder Internet von mehreren Spielern gemeinsam
gespielt wird. Die Avatare der Spieler befinden sich in einer gemeinsamen virtuellen Welt. Diese Welt wird von der
Spielmechanik so synchronisiert, dass alle Spieler Ereignisse und Ver�nderungen darin in gleicher Weise wahrnehmen
k�nnen. Gegner und Teammitglieder werden teilweise oder ausschlie�lich von realen Mitspielern gesteuert, was trotz der
stetig voranschreitenden KI-Entwicklung auch heute noch ein dynamischeres und realistischeres Verhalten der virtuellen
Spielfiguren erzeugt als bei sog. "BOTs" (engl. kurz f�r "robot"), vom Computer simulierten Mitspielern. 

\subsection{Weitere Abgrenzung}
Das Spiel l�uft in Runden mit stets der gleichen Ausgangssituation ab. Auf eine Handlung im Sinne eines Drehbuchs (wie
bei Singleplayer-Spielen �blich) sowie auf eine fortgesetzte Entwicklung der Spielwelt sowie des Charakters/Avatars (wie
bei Singleplayer- und Rollenspielen �blich) wird weitestgehend verzichtet.\\

Die Spielerzahl liegt in einem �berschaubaren Bereich. �blicherweise zwischen 10 und 64 Spielern pro Instanz, aber auch
Spiele mit bis zu 128 oder gar noch mehr Spielern sind bereits zu finden. Dennoch deutlich ist der Unterschied zu sog.
MMOs, bei denen oft mehrere tausend Spieler in einer Instanz spielen.\\

Bekannte Multiplayer-Taktik-Shooter im Sinne dieser Definition sind z.B. Counterstrike, Battlefield und Call of Duty.
Nicht darunter fallen Singleplayer-Spiele wie Tomb Raider und Splinter-Cell oder Shooter, denen der Taktik-Anteil fehlt,
wie z.B. Quake 3 Arena.\\