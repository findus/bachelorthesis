\chapter{Einf�hrung}
\label{chapter:Introduction}

Seit langem herrscht in Politik und Gesellschaft Uneinigkeit �ber die Auswirkungen von Gewaltdarstellungen in
Computerspielen. Trotzdem sind solche stets in einigen Spielen enthalten gewesen. Besonders das Genre der Shooter ist f�r
seine detaillierten und intensiven Gewaltdarstellungen bekannt, der Anteil an Spielen mit hohen Altersbeschr�nkungen ist
darin besonders hoch (siehe Abb. \ref{figure:stigmafreigaben}). Dennoch definiert sich dieses Genre - entgegen der oft kritischen
�ffentlichen Meinung - nicht, oder zumindest nicht ausschlie�lich, �ber die enthaltenen Gewaltdarstellungen.

\begin{figure}[htbp]
\centering
\includegraphics[width=0.8\textwidth]{images/stigmafreigaben}
\caption[Statistik der USK-Altersfreigaben nach Genres seit 1994]{Statistik der USK-Altersfreigaben nach Genres seit 1994\footnotemark}
\label{figure:stigmafreigaben}
\end{figure}
\footnotetext{Quelle: http://stigma-videospiele.de/wordpress/rechtslage/statistiken/}

\section{Zielsetzung}
Ziel dieser Arbeit ist die Analyse des Genres Multiplayer-Taktik-Shooter und die Erarbeitung eines Konzepts f�r einen
gewaltfreien Genrevertreter. Es soll gezeigt werden, dass das Spielprinzip des Shooters auch ohne Gewaltdarstellungen,
speziell ohne Schie�en auf und T�ten von Gegnern, funktioniert. Dazu sollen zum Einen die Eigenschaften, in denen sich
der Shooter �ber das Schie�en hinaus von anderen Genres unterscheidet und damit selbst definiert, identifiziert und
verwendet werden. Zum Anderen sollen M�glichkeiten f�r neue Spielelemente gefunden und untersucht werden, die eine
eventuell entstehende L�cke im Konzept durch den Wegfall des Elements "`Schie�en"' f�llen oder ausgleichen k�nnen.\par

Es soll also gezeigt werden, dass (und wie) ein gutes Spiel entwickeln werden kann, das alle Eigenschaften und Elemente des
Shooters hat, die mit dem gewaltfreien Grundsatz vereinbar sind, und somit das gleiche Klientel zufriedenstellt.

\begin{center}
Shooter - Schie�en + X $\rightarrow$ Spielqualit�t
\end{center}

Dabei ist mit Spielqualit�t Spielspa�, Motivation und Schwierigkeit gemeint, die an den Erwartungen von Shooterspielern
gemessen werden. Der Platzhalter X steht f�r Spielemente, die neu hinzugef�gt werden. Solche sollen in dieser Arbeit
gefunden und erprobt werden.
