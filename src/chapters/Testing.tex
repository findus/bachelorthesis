\chapter{Testing}
\label{chapter:Testing}


\section{Anwendertests}

Mehrfach w�hrend der Entwicklung wurde das Spiel Testpersonen vorgesetzt. Diese sollten ohne weitere Vorbereitung spielen und ihre Erfahrungen dann mitteilen.
Dabei wurden viele wesentliche Probleme und Schwachstellen aber auch positive und erhaltenswerte Elemente identifiziert. Die wichtigsten davon sollen im folgenden erl�utert werden.

Der im Rahmen dieser Arbeit entwickelte Spiele-Prototyp hatte zu einem fr�hen Entwicklungszeitpunkt folgenden
Spielablauf: Die Spielwelt bestand aus Bauklotz-�hnlichen W�rfeln, die die Spieler beliebig umbauen konnten. Ziel war es
ausgehend von einem Startpunkt als erster einen Zielpunkt zu erreichen. Dazu mussten Treppen, Br�cken und andere
Konstruktionen gebaut werden. Im Test mit mehreren Spielern zeigte sich, dass die M�glichkeit, dem Gegner W�nde in den
Weg zu bauen oder dessen Konstruktionen wieder zu zerlegen, nicht ausreichte um die Spieler zu motivieren. Ein Spieler
�u�erte sich folgenderma�en: "Mir fehlt was, um den anderen abzuknallen." Ebenso bot sich den Spielern kaum Raum,
kooperativ vorzugehen, was sich an der Aussage eines anderen Spielers erkennen l�sst: "Nein, ich wusste nicht, dass wir
im gleichen Team sind".