\section{Testing}

Im Rahmen dieser Arbeit wurden nicht nur das Problem der Gewaltdarstellungen in Shootern untersucht und ein Konzept
erarbeitet, wie Shooter auch ohne diese auskommen k�nnen, sondern auch ein Prototyp eines gewaltfreien Shooters
entwickelt. Das hat im wesentlichen den Zweck, alte und neue Ideen und Konzepte leicht testen zu k�nnen. Ein auf dem
Papier oder im Kopf konstruiertes Spielelement offenbart im implementierten Einsatz bei Testspielern g�nzlich neue
Seiten. Nur so kann ein entwickeltes Spielkonzeptes ernsthaft validiert werden.\\

\subsection{Usability Tests}

Mehrfach w�hrend der Entwicklung wurde der Prototyp Testpersonen vorgesetzt. Diese sollten ohne weitere Vorbereitung
spielen und ihre Erfahrungen dann mitteilen. 

Der Prototyp hatte zu einem fr�hen Entwicklungszeitpunkt folgenden Spielablauf: Die Spielwelt bestand bereits aus den
"`Cubes"' genannten Bauklotz-�hnlichen W�rfeln, die die Spieler beliebig umbauen konnten. Ziel war es ausgehend von
einem Startpunkt als erster einen farblich hervorgehobenen Zielpunkt zu erreichen. Dazu mussten Treppen, Br�cken und
andere Konstruktionen gebaut werden. Im Test mit mehreren Spielern zeigte sich, dass die M�glichkeit, dem Gegner W�nde
in den Weg zu bauen oder dessen Konstruktionen wieder zu zerlegen, nicht ausreichte um die Spieler zu motivieren. Ein
Spieler �u�erte sich folgenderma�en: "`Mir fehlt was, um den anderen abzuknallen."' Ebenso bot sich den Spielern kaum
Raum, kooperativ vorzugehen, was sich an der Aussage eines anderen Spielers erkennen l�sst: "`Nein, ich wusste nicht,
dass wir im gleichen Team sind"'.

%Meine Freundin ist ein Weinbeer, ein armes trauriges Beerchen. Aber KEIN Pomb�r!!! Vernaschen kann ich sie trotzdem. Manchmal! Glaaaabst! Jawooohhhl!
\subsubsection{Lanparty}

Auf einer Lanparty wurde den Spielern eine Entwicklungsversion des Prototypen zur Verf�gung gestellt, damit sie das
Spiel ausprobieren. Es waren etwa 20 Spieler anwesend, jeder mit seinem eigenen PC. Die Spieler konnten die Anwendung
�ber Java-Webstart von einem lokalen Web-Server starten. Nachdem anf�ngliche Schwierigkeiten beim Starten des Spiels,
konnten sieben Spieler gemeinsam in einer Instanz spielen. Die Features in dieser Version umfassten einen Serverbrowser,
Teamwahlmen�, Pausemen� mit Beenden-Button. Als Equipments standen dem Spieler Pick'n'Place sowie der Beamer zur
Verf�gung.
Folgende Ergebnisse lassen sich festhalten:
\begin{itemize}
\item 1. Da die automatische Serversuche (mittels Multicast) nicht auf allen PCs funktioniert, m�ssen viele Spieler die IP-Adresse des Servers manuell eingeben, was sie offensichtlich �berfordert.
\item 2. Die Steuerung ist nicht f�r alle Spieler intuitiv genug. Umsehen mit der Maus ist f�r alle Spieler bekannt. Bewegen mit WASD ebenso. Springen mit Shift versuchen nur die wenigsten, die meisten erwarten die Leertaste zum Springen. 
\item 3. Das Icon f�r das aktuelle Equipment rechts unten auf dem Bildschirm erkl�rt nicht ausreichend die Funktionsweise des Equipments.
\item 4. Die 
\end{itemize}