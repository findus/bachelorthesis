\section{Case Study}
\label{CaseStudy}

\subsection{Orientierung an vohandenen Spielen}
Da ein Vertreter der Multiplayer-Taktik-Shooter entstehen soll, liegt es nat�rlich nahe, bereits existierende Spiele des
Genres als Vorlage heranzuziehen. Von diesen werden besonders die bereits genannten Counterstrike und Battlefield
betrachtet.\\

Desweiteren bietet es sich an, Spiele zu untersuchen, die Gemeinsamkeiten bez�glich gewisser Eigenschaften oder
Spielelemente mit den oben genannten Spielen aufweisen. Darunter fallen zum Beispiel eine relativ junge Form von (meist
physikbasierten) R�tselspielen, wie z.B. Portal oder Qube, die die Ego-Perspektive und die Steuerung der Shooter �bernommen haben.\\

Auch reine Multiplayer-Shooter (also solche, die nicht als Taktik-Shooter bezeichnet werden k�nnen), die neben dem T�ten
m�glichst vieler Gegner jedoch ein alternatives Spielziel anbieten, wie zum Beispiel Unreal Tournament im "`Capture the
Flag"'-Modus, sollen untersucht werden.

\subsection{Eigenschaften existierender Multiplayer-Taktik-Shooter}
Bei der Untersuchung bekannter Vertreter des Genres fallen vor allem Eigenschaften auf, die bereits in der Genre-Abgrenzung in Kapitel
\ref{multiplayertacticshooter} erw�hnt wurden. Diese und einige weitere sind in der folgenden Liste aufgef�hrt.
\begin{itemize} 
\item Ego-Perspektive
\item Steuerung mit Maus(Umsehen, Zielen) und Tastatur (Bewegen)
\item Multiplayer (Server-Client-Modell)
\item Zwei Teams, die als Gegner antreten
\item Gameplay, das Kooperation ebenso wie K�nnen und �bung belohnt
\item Runden mit stets gleicher Ausgangssituation
\item Rundenende nach bestimmter Zeit oder Erreichen eines Spielziels
\item ausw�hlbare unterschiedliche Spielwelten (sog. Karten)
\end{itemize}

\subsection{Spielmodi}
Als Spielmodus werden in der Literatur verschiedene Arten der Kategorisierung von Spielen angesehen. Im folgenden soll
unter Spielmodus das konkrete Regelwerk des Spiels verstanden werden, also das Ziel bzw. die Ziele des Spiels, sowie die
Handlungen, die zum Erreichen eines Ziels erforderlich sind. Wenn man in Anlehnung an die Topologie f�r Spiele von
Aarseth, Smedstad und Sunnan� \cite{Aarseth_Smedstad_Sunnan�_2003} vierzehn Dimensionen zur Typologisierung von
Computerspielen verwendet, k�nnen Perspektive, Topographie, Zeitlicher Ablauf, Spielerstruktur, Saveability,
Zuf�lligkeit ??? als Teil der Genre-Einschr�nkung gesehen werden. Die �brigen (Umgebung, Teleologie, Teamstruktur,
Charakter-Ver�nderung, Topographische Regeln, Zeitbasierte Regeln, Aufgabenbasierte Regeln ???) entscheiden �ber den
Spielmodus. Bei diesen Dimensionen m�ssen jedoch nicht nur deren m�gliche Werte nach Aarseth, Smedstad und Sunnan�
betrachtet werden (z.B. Aufgabenbasierte Regeln: ja, nein), sondern die tats�chlichen Ausf�hrungen derselben.

Perspective: Onmi-present, Vagrant
Perspektive: alles-�berblickend, wandernd

Topography: Geometrical, topological
Topographie: kontinuierlich, diskret

Environment: Dynamic, Static
Umgebung: dynamisch, statisch

Pace: Realtime, Turnbased
Zeitlicher Aufbau: Echtzeit, Rundenbasiert

Representation: Mimetic, Arbitrary
Repr�sentation: mimetisch, beliebig

Teleology: finite, infinite
Teleologie: endlich, unendlich

Playerstructure: Singleplayer, twoplayer, multiplayer, singleteam, twoteam, multiteam
Spielerstruktur: Einzelspieler, Zweispieler, Mehrspieler

Teamstruktur: Einzelteam, Zweiteam, Mehrteam

Mutability: static, powerups, experience-leveling (XL)
Charakter-Ver�nderung: statisch, vor�bergehend, progressiv

Savability: non-saving, conditional, Un-limited
Speicherbarkeit: kein, bedingt, jederzeit

Determinism: deterministic, non-deterministic
Zuf�lligkeit: deterministisch, zuf�llig

Topologicalrules: yes, no
Topographische Regeln: ja, nein

Timebasedrules: yes, no
Zeitbasierte Regeln: ja, nein

Objectivebased rules: yes, no
Aufgabenbasierte Regeln: ja, nein

