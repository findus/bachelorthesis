\chapter{System Design}
\label{chapter:SystemDesign}

\section{Wahl der Programmiersprache}

Im Bereich der Computerspiele wird heute �berwiegend C++ eingesetzt. Hauptgrund daf�r ist die hohe Performance, au�erdem
die OO-Unterst�tzung und die Tatsache, dass die meisten Tools und Bibliotheken auf C++ basieren oder dazu kompatibel
sind. Da das Ziel dieser Arbeit allerdings nicht die Entwicklung eines marktreifen AAA-Games, sondern vielmehr eines
Prototypen war, konnte der Performance-Faktor weitgehend vernachl�ssigt werden. Bereits vorhandene Erfahrung im Umgang
mit der Programmiersprache, den verf�gbaren Libraries und der Entwicklungsumgebung, sowie die hohe Produktivit�t der
Sprache sind entscheidend f�r die Wahl von Java verantwortlich.

\section{Wahl der Engine}

\subsection{Was ist eine Game-Engine?}
Ein Computerspiel besteht meist aus einer Spielewelt, die optisch und aktustisch erlebt werden kann. Au�erdem kann mit
dieser Welt - je nach Spiel auf unterschiedliche Weise - interagiert werden. In vielen Spielen steuert der Spieler
seinen Hauptcharakter, den sog. Avatar, durch diese Welt. Dazu werden Softwarebibliotheken ben�tigt, die
dreidimensionalen Raumklang erzeugen, Objekte grafisch darstellen, physikalische Abl�ufe berechnen, Benutzereingaben in
Steuerbefehle umwandeln, sowie Ereignisse �ber das Netzwerk kommunizieren. Eine Spiele-Engine bietet dem Entwickler
diese und andere Funktionalit�ten �ber ein einheitliches Interface an. Zum einen wird dadurch der Programmcode besser
strukturiert, zum anderen der Entwickler entlastet.
Es bietet sich dabei nat�rlich an, eine Engine f�r mehrere (�hnliche) Spiele zu verwenden. Einige Hersteller stellen
ihre Engine anderen Entwicklern (teils gegen Lizenzgeb�hren) zur Verf�gung. Bekannte Engines gro�er Spielehersteller
sind die Frostbite-Engine (Dice), Cry-Engine (Crytec), Unreal-Engine (Epic), die neben dem Hauptspiel, f�r das sie
entwickelt wurden, auch in weiteren Spielen verwendet wurden.

Es ist nat�rlich m�glich, ein Spiel ohne bereits existierende Game-Engine zu schreiben. Das f�hrt allerdings in den
meisten F�llen dazu, dass man eine eigene Engine innerhalb des Spiels entwickelt. Da heutzutage reichlich gute Engines
verf�gbar sind, einige davon sogar ohne Lizenzgeb�hren frei verwendbar, macht dieser Mehraufwand wenig Sinn. Vor allem
ist zur Entwicklung einer guten Engine viel Erfahrung n�tig, die die wenigsten Entwickler haben d�rften.

\subsection{Game-Engine f�r Java}
Die meisten Game-Engines sind in C/C++ geschrieben. Eine solche Engine als Teil einer Java Software zu verwenden w�re
�ber das JNI evtl. m�glich, ist aber sicher keine einfache L�sung. Zum Gl�ck gibt es auch einige Engines die zu Java
kompatibel oder gar in Java geschrieben sind.
Die Auswahl ist nicht sehr gro�, so l�sst sich etwa in der Auflistung der Wikipedia schnell die Java MonkeyEngine als
bestes Gesamtpaket identifizieren. Sie bietet alle n�tigen Features\footnote{Liste aller Features der jMonkeyEngine3:
http://jmonkeyengine.org/engine/}, ist schon in der dritten Version verf�gbar, somit ist von einem Gewissen Reifegrad
auszugehen, sie ist frei verwendbar unter der BSD-Lizenz und sie rendert mithilfe der Java-OpenGL-Anbindung
JOGL\footnote{siehe: https://jogamp.org/jogl/www/}, was eine gewisse Grafikperformance erwarten l�sst.


\section{Architektur}

Das Spiel besteht Serverkomponente, die f�r die Berechnung der virtuellen Welt und der Spielabl�ufe verantwortlich ist,
sowie einer Clientkomponente, welche die virtuelle Welt audiovisuell darstellt und Benutzereingaben verarbeitet.

\section{Hardware-Software Mapping}

Das Spiel kann entweder im Internet oder im lokalen Netzwerk gespielt werden. Im ersten Fall l�uft die Serversoftware
auf einem gehosteten Server im Internet, der per IP-Adresse oder Domain erreichbar ist. Im zweiten Fall l�uft die
Server-Software auf einem der PCs der Spieler oder einem extra daf�r eingerichteten zus�tzlichem Rechner/Server. Die
Clientsoftware wird in beiden F�llen auf dem Computer jedes Spielers gestartet. Technisch ist es m�glich, beliebig viele
Instanzen der Clientsoftware auf einer Maschine laufen zu lassen, jedoch stets nur eine Instanz der Server-Software, da
diese festgelegte Ports reserviert.