\section{Game Design}

In diesem Kapitel soll aus anhand der bisher gewonnenen Erkenntisse ein Spiel-Konzept erstellt werden, das den
gewaltfreien Ansatz verwirklicht und dabei die gew�nschten Shooter-Elemente enth�lt. Um Verwirrungen zu vermeiden,
erh�lt das neue Spiel bereits an dieser Stelle einen Namen, sozusagen einen Entwicklungs-Codenamen: Cydonia.


\subsection{Grundlegende Eigenschaften}
Cydonia soll ein Multiplayerspiel f�r ca. 5 bis 25 Spieler werden. Vorerst ist nur eine Unterst�tzung f�r
Local-Area-Network (LAN) n�tig, eine zuk�nftige Unterst�tzung von Spielen �ber das Internet ist jedoch nicht auszuschlie�en.

Der Spieler soll seinen Avatar aus der Ego-Perspektive mittels Maus und Tastatur steuern k�nnen.

Cydonia bekommt ein futuristisches Setting, dh. die Spielwelt sowie die Charaktere sind nicht der realen Gegenwart oder
Vergangenheit entnommen, sondern in einer fiktiven Zukunft angesiedelt. Dadurch wird zum einen die Schwierigkeit
umgangen, reale Lebewesen und Gegenst�nde �berzeugend und nat�rlich abzubilden, zum anderen bietet ein fiktives Setting
mehr kreativen Freiraum als ein reales, da unwahrscheinliche oder unm�gliche Dinge, Vorg�nge, Zusammenh�nge auf
den Spieler weniger irritierend wirken, wenn die gesamte Spielwelt bereits unrealistisch anmutet.


\subsection{Spielwelt}
Die virtuelle Welt von Cydonia besteht fast ausschlie�lich aus farbigen W�rfeln mit einem Meter Kantenl�nge. Diese
W�rfel bilden B�den, W�nde und D�cher. Die Spieler k�nnen eine W�rfelh�he �berspringen, wodurch auch Treppen m�glich
werden. Zur besseren semantischen Unterscheidung dieser W�rfel von der geometrischen Form werden die W�rfel der
Spielwelt im folgenden "`Cube"' genannt.

\subsection{Spielmodus}
Cydonia bedient sich bei einem bekannten Modus, den bereits manche andere Shooter (z.B. Unreal Tournament)
anbieten: "`Capture the Flag"'. Dabei versuchen zwei Teams von Spielern, eine Flagge aus der gegnerischen in die eigene
Basis zu bringen und gleichzeitig das gegnerische Team von der eigenen Flagge fernzuhalten.


\subsection{Swapper}
Ein Equipment\footnote{vlg. dazu \ref{CaseStudy}} soll in Cydonia die Funktion der Waffen in anderen Shootern
�bernehmen, dem Spieler also eine direkte Einwirkung auf den Gegner erm�glichen. Mit dem sog. "`Swapper"' kann der
Spieler Cubes und andere Spieler markieren. Sobald er zwei Dinge markiert hat, tauschen diese beiden den Platz.
Dadurch kann der Spieler also seine Gegner von ihrem Pfad abbringen und in der Erf�llung ihres Auftrags behindern.
Gleichzeitig bietet sich aber auch die M�glichkeit, durch geschicktes "`Swappen"' von Mitgliedern des eigenen Teams
kooperativ zu agieren. Eine dritte Art der Verwendung besteht darin, durch das Vertauschen von Teilen der Spielwelt, das
Leveldesign zu seinem Vorteil zu ver�ndern.


\subsection{Pick'n'Place}
Der Spieler bekommt in Cydonia ein weiteres Equipment in die Hand, das nicht auf Gegner, sondern ausschlie�lich auf
Cubes angewendet werden kann. Er kann damit einzelne Cubes in seiner Umgebung entfernen und (an anderer Stelle) wieder
einf�gen und so das Leveldesign w�hrend des Spiels beeinflussen. 


\subsection{Nahkampf}
F�r das in Kapitel \ref{CaseStudy} erl�utertete Spielelement Nahkampf konnte bisher keine gewaltfreie Alternative
gefunden werden, weshalb Cydonia vorerst auf ein solche Element verzichten muss.
