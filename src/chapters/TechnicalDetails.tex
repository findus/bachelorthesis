\chapter{Technische Details}
\label{chapter:TechnicalDetails}


\section{Wahl der Programmiersprache}

Im Bereich der Computerspiele wird heute �berwiegend C++ eingesetzt. Hauptgrund daf�r ist die hohe Performance, au�erdem
die OO-Unterst�tzung und die Tatsache, dass die meisten Tools und Bibliotheken auf C++ basieren oder dazu kompatibel
sind. Da das Ziel dieser Arbeit allerdings nicht die Entwicklung eines marktreifen AAA-Games, sondern vielmehr eines
Prototypen war, konnte der Performance-Faktor weitgehend vernachl�ssigt werden. Bereits vorhandene Erfahrung im Umgang
mit der Programmiersprache, den verf�gbaren Libraries und der Entwicklungsumgebung, sowie die hohe Produktivit�t der
Sprache sind entscheidend f�r die Wahl von Java verantwortlich.

\section{Wahl der Engine}

\subsection{Was ist eine Game-Engine?}

Ein Computerspiel besteht meist aus einer Spielewelt, die optisch und aktustisch erlebt werden kann. In vielen Spielen
steuert der Spieler seinen Hauptcharakter, den sog. Avatar, durch diese Welt. Dazu werden Softwarebibliotheken ben�tigt,
die dreidimensionalen Raumklang erzeugen, Objekte grafisch darstellen, physikalische Abl�ufe berechnen, Benutzereingaben
in Steuerbefehle umwandeln, sowie Ereignisse �ber das Netzwerk kommunizieren. Eine Spiele-Engine bietet dem Entwickler
diese und andere Funktionalit�ten �ber ein einheitliches Interface an. Zum einen wird dadurch der Programmcode besser
strukturiert, zum anderen der Entwickler entlastet.
Es bietet sich dabei nat�rlich an, eine Engine f�r mehrere (�hnliche) Spiele zu verwenden. Einige Hersteller stellen
ihre Engine anderen Entwicklern (teils gegen Lizenzgeb�hren) zur Verf�gung. Bekannte Engines gro�er Spielehersteller
sind die Frostbite-Engine (Dice), Cry-Engine (Crytec), Unreal-Engine (Epic), die neben dem Hauptspiel, f�r das sie
entwickelt wurden, auch in weiteren Spielen verwendet wurden.

Es ist nat�rlich m�glich, ein Spiel ohne bereits existierende Game-Engine zu schreiben. Das f�hrt allerdings in den
meisten F�llen dazu, dass man eine eigene Engine innerhalb des Spiels entwickelt. Da heutzutage reichlich gute Engines
verf�gbar sind, einige davon sogar ohne Lizenzgeb�hren frei verwendbar, macht dieser Mehraufwand wenig Sinn. Vor allem
ist zur Entwicklung einer guten Engine viel Erfahrung n�tig.

\subsection{Game-Engine f�r Java}

Die meisten Game-Engines sind in C/C++ geschrieben. Eine solche Engine als Teil einer Java Software zu verwenden w�re
�ber das JNI evtl. m�glich, ist aber sicher keine einfache L�sung. Zum Gl�ck gibt es auch einige Engines die zu Java
kompatibel oder gar in Java geschrieben sind.
Die Auswahl ist nicht sehr gro�, so l�sst sich zum etwa in der Auflistung der Wikipedia schnell die Java MonkeyEngine
bestes Gesamtpaket identifizieren.