\chapter{Ersatz f�r Gewaltdarstellungen}
\label{chapter:ViolenceSubstitution}


Ein Ansatz, bei den klassischen Shootern, die als Vorlage f�r das neue Spiel dienen sollen, Gewaltfreiheit zu erzwingen,
indem s�mtliche Gewaltdarstellungen entfernt werden, erweist sich als problematisch. Ein in dieser Art gek�rztes Spiel
w�re inhaltslos und somit langweilig, verliert also letztlich seine Daseinsberechtigung. Um den Spielwert aufrecht zu
erhalten, m�ssen also die Gewaltdarstellungen ver�ndert oder durch andere Spielelemente ersetzt werden. In typischen
R�tselspielen, Simulationen oder Strategiespielen wird der Spieler meist auch ohne Gewaltdarstellungen
unterhalten. Warum das bei Ego-Shootern - besonders im Multiplayerbereich - so schwierig ist, wird im folgenden
Abschnitt untersucht.


\section{Perspektive}

 
\section{Identifikation der Spielelemente}
Um die Gewaltdarstellungen aus dem Spiel entfernen zu k�nnen, m�ssen zuerst alle betroffenen Spielelemente identifiziert
werden. Dies soll im folgenden anhand bekannter Genrevertreter geschehen.

Hauptbestandteil eines Shooters ist - wie der Name schon erkennen l�sst - das Schie�en. Dies wird dem Spieler
�blicherweise durch ein Arsenal unterschiedlicher Waffen erm�glicht. W�hrend zum Beispiel Counterstrike reale
Waffenmodelle wie die AK47 oder die DesertEagle nachbildet, kommen in UnrealTournament futuristische Waffen wie
Laserkanonen zum Einsatz. Beiden gemeinsam ist, dass der Spieler mit diesen Waffen seine virtuellen Gegner durch
Beschuss verletzen und letztendlich t�ten kann. Ein get�teter Gegner scheidet - meist vor�bergehend - aus dem Spiel aus,
der Sch�tze bekommt Punkte. Neben der detaillierten Abbildung von Waffe und unter Umst�nden auch Projektil wird auch der
Tod des Gegners animiert, teils sehr blutig.

Abstrakter gesehen ist das Schie�en eine direkte Einwirkung auf den Gegner. Wesentliche Merkmale sind das Zielen auf die
virtuelle Spielfigur, das Aktivieren durch Tastendruck, sowie die negative, sch�digende und den Gegner in seinem
Spielablauf behindernde Wirkung. Ein hohes K�nnen (sog. Skill) erleichtert bzw. erm�glicht dem Spieler also das
Erreichen seines Spielziels, w�hrend es gleichzeitig dem Gegner das Erreichen seines Spielziels erschwert bzw. unm�glich
macht.

