\chapter{Konzept}
\label{chapter:concept}

Die Konzepterstellung verfolgt zwei Ziele: Zum einen sollen die Eigentschaften, in denen sich der Shooter au�er dem
Schie�en noch von anderen Genres unterscheidet und damit selbst definiert, identifiziert und verwendet werden. Zum
Anderen sollen M�glichkeiten f�r neue Spielelemente gefunden und untersucht werden, die eine evetnuell entstehende
L�cke im Konzept durch den Wegfall des Elements "Schie�en" f�llen oder ausgleichen k�nnen.\\

Die These hinter dieser Aufgabenstellungen l�sst sich also zusammenfassen als:

\begin{center}
Shooter - Schie�en + X => Spielqualit�t
\end{center}

Dabei ist mit Spielqualit�t Spielspa�, Motivation und Schwierigkeit gemeint, die an den Erwartungen von Shooterspielern
gemessen werden.


\section{Elemente des Shooters}
Die meisten Eigenschaften des Shooters, die in die Entwicklung einflie�en sollen, lassen sich der obigen
Genre-Eingrenzung entnehmen. Weitere Elemente werden aus bereits existierenden Vertretern des Genres �bernommen, zum
Beispiel aus Counterstrike oder Battlefield. Daraus entstehen folgende Anforderungen an das neue Spiel:
\begin{itemize} 
\item Ego-Perspektive
\item Steuerung mit Maus(Umsehen, Zielen) und Tastatur (Bewegen)
\item Multiplayer (Server-Client-Modell)
\item Zwei Teams, die als Gegner antreten
\item Gameplay, das Kooperation ebenso wie K�nnen und �bung belohnt
\item Runden mit stets gleicher Ausgangssituation
\item Rundenende nach bestimmter Zeit oder Erreichen eines Spielziels
\item ausw�hlbare unterschiedliche Spielwelten (sog. Karten)
\end{itemize}


\section{Spielziel}
Da das T�ten von Gegnern als Spielziel nicht zur Verf�gung steht, wird ein anderes ben�tigt. Counterstrike l�sst die
Teams zum Beispiel durch Bewachen/Befreien von Geiseln bzw. durch Legen/Entsch�rfen einer Bombe gewinnen. Beide
Alternativen sind jedoch f�r ein gewaltfreies Spiel nicht geeignet.\\

Der Versuch, einen Spielmodus zu finden oder zu entwickeln, scheitert an einem Problem, das fast alle denkbaren
Spielmodi betrifft. Ohne eine M�glichkeit neben dem Bestreben, das eigene Spielziel m�glichst schnell zu erreichen, auch
den Gegner auf dem Weg zu seinem Spielziel aufzuhalten, verliert ein Multiplayer-Spiel seinen interaktiven Charakter und
verkommt dann zu einem reinen Wettrennen. \\


Der im Rahmen dieser Arbeit entwickelte Spiele-Prototyp hatte zu einem fr�hen Entwicklungszeitpunkt folgenden
Spielablauf: Die Spielwelt bestand aus Bauklotz-�hnlichen W�rfeln, die die Spieler beliebig umbauen konnten. Ziel war es
ausgehend von einem Startpunkt als erster einen Zielpunkt zu erreichen. Dazu mussten Treppen, Br�cken und andere
Konstruktionen gebaut werden. Im Test mit mehreren Spielern zeigte sich, dass die M�glichkeit, dem Gegner W�nde in den
Weg zu bauen oder dessen Konstruktionen wieder zu zerlegen, nicht ausreichte um die Spieler zu motivieren. Ein Spieler
�u�erte sich folgenderma�en: "Mir fehlt was, um den anderen abzuknallen." Ebenso bot sich den Spielern kaum Raum,
kooperativ vorzugehen, was sich an der Aussage eines anderen Spielers erkennen l�sst: "Nein, ich wusste nicht, dass wir
im gleichen Team sind".\\

Nahkampf\\




\section{Orientierung}
Da ein Vertreter der Multiplayer-Taktik-Shooter entstehen soll, liegt es nat�rlich nahe, bereits existierende Spiele des
Genres als Vorlage heranzuziehen. Von diesen werden besonders die bereits genannten Counterstrike und Battlefield
betrachtet.\\

Desweiteren bietet es sich an, Spiele zu untersuchen, die Gemeinsamkeiten bez�glich gewisser Eigenschaften oder
Spielelemente mit den oben genannten Spielen aufweisen. Darunter fallen zum Beispiel eine relativ junge Form von (meist
physikbasierten) R�tselspielen, wie z.B. Portal oder Qube, die die Ego-Perspektive und die Steuerung der Shooter �bernommen haben.\\

Auch Shooter, die ein neben dem T�ten m�glichst vieler Gegner ein alternatives Spielziel anbieten, wie zum Beispiel
Unreal Tournament im "Capture the Flag"-Modus, sollen untersucht werden.
