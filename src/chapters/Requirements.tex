\section{Anforderungen}
\label{requirements}

\subsection{Funktionale Anforderungen}

\subsubsection{FR 1: Darstellung der virtuelle Welt}
Das Spiel muss die virtuelle Spielwelt, bestehend aus farbigen W�rfeln, Flaggenpunkten und Spielern grafisch
dreidimensional darstellen. Der Anwender soll dabei die virtuelle Welt aus der Perspektive seines Avatars sehen.

\subsubsection{FR 2: Auswahl der Spielwelt}
Das Spiel muss dem Anwender die M�glichkeit bieten, unter verschiedenen Spielwelten, sog. Karten, zu w�hlen.

\subsubsection{FR 3: Steuerung}
Das Spiel muss Eingaben des Anwenders via Maus und Tastatur zur Steuerung des Avatars umsetzen. Dazu geh�rt:
\begin{itemize}
  \item Bewegen in vier Richtungen
  \item Springen
  \item Umsehen in alle Richtungen
  \item Wechseln des Equipments
  \item Benutzen des Equipments
\end{itemize}

\subsubsection{FR 4: Ablauf}
Das Spiel muss rundenweise ablaufen. Die Ausgangssituation der Spielwelt sowie der Avatare muss jede Runde gleich sein.
Jede Runde endet nach einer bestimmten Zeit oder, wenn ein Team eine bestimmte Anzahl Punkte erreicht.
Ausschlaggebend ist das Ereignis, das fr�her eintritt.


\subsection{Nicht-funktionale Anforderungen}
\subsubsection{Usability}
Im Bereich der Spiele-Entwicklung k�nnendie nach ISO-Standart �blichen Ma�st�be f�r Usability, Effektivit�t (die
Genauigkeit und Vollst�ndigkeit mit der Benutzer ihre gesetzten Ziele erreichen), Effizienz (die Ressourcen, die zur
Erf�llung der Vorgaben eingesetzt werden) und die Zufriedenheit des Benutzers, nicht auf die gesamte Software (in diesem
Fall das Spiel) angewendet werden. Da ein Spiel m�glichst lange unterhalten soll, ist Effizienz, im Sinne m�glichst
schneller Erledigung, hinf�llig. Au�erdem haben Spiele oft keinen definierten Endpunkt, an dem das Spiel als
abgeschlossen gilt. Wenn sie einen solchen Endpunkt besitzen, gibt es oft viele Wege, diesen zu
erreichen.\cite{Federoff_2002} Man kann allerdings die �bliche Heuristik durchaus auch auf Spiele anwenden, wenn man
lediglich das Interface betrachtet.\\

Vor allem in schnellen Spielen, zu denen Shooter tendenziell immer geh�ren, ist es f�r den Spieler wichtig, alle im
Spielfluss ben�tigten Funktionen per einfachen Tastendruck zu erreichen. Nicht umsonst ist bei der f�r Shooter �blichen
Steuerung Zielen und Abfeuern der Waffe in einer Hand, der "`Maushand"', vereint. Gleichzeitig �bernimmt die andere Hand
alle f�r die Bewegung des Avatars n�tigen Eingaben: vorw�rts, r�ckw�rts, seitw�rts, springen, ducken usw.. Seltener
gebrauchte Funktionen werden meist ebenso mit schlechter erreichbaren Tasten verbunden, wie Funktionen, bei denen die
Reaktionsgeschwindigkeit nicht allzu spielentscheidend ist.\\

Zur Usability geh�rt aber nicht nur die schnelle und einfach Eingabe von Befehlen, sondern auch die �bersichtlichkeit
der grafischen Anzeige. Da Spieler in dreidimensionalen Spielen stark auf die Darstellung der virtuellen Welt fokusiert
sind, m�ssen zus�tzliche Informationen (wie �brige Zeit, Punktestand, Position der Teammitglieder, \ldots) jederzeit
�bersichtlich und mit starkem Kontrast zum Hintergrund dargestellt werden, sodass der Spieler mit einem kurzen Blick
alle n�tigen Imformationen erhalten kann.\\

Und schlie�lich ist auch in Spielen die einfache Bedienbarkeit aller Men�s von Bedeutung. Dazu m�ssen die einzelnen
Optionen sinnvoll gruppiert und mit m�glichst wenigen Klicks erreichbar sein.

\subsubsection{Security}
Da es sich um ein Multiplayer-Spiel handelt, muss gew�hrleistet werden, dass kein Spieler sich auf unfaire Weise einen
Vorteil verschaffen kann.

\subsubsection{Performance}
Performance bezieht sich in diesem Fall auf das Simulieren (u.a. Physikberechnung) und Rendern der virtuellen Welt, da
diese typischer Weise die Flaschenh�lse in der Spiele-Software sind. Um dem Spieler ein fl�ssiges Spielerlebnis zu
erm�glichen und die Immersion nicht zu st�ren, sollte die Bildwiederholrate nie unter 30 fps (Frames per second) fallen.
