\section{Szenarien}
\label{section:Scenarios}

"`Ein Szenario ist die konkrete, fokussierte und informelle Beschreibung eines einzelnen Systemmerkmals aus der Sichtweise
eines einzelnen Benuzters."'\cite[S. 157]{Bruegge_2004}
Die folgenden Szenarien beschreiben einige Situationen, die bei der Benutzung des neuen Spiels vorkommen k�nnten.

\subsection{Hans der Br�ckenbauer}
Hans spielt Cydonia. Er ist in Team 2. Er spawnt neben der Flagge seines Teams und l�uft in Richtung der Flagge von Team
1. Auf seinem Weg dorthin kommt er an ein gro�es Loch im Boden, �ber das er nicht springen kann. Er sieht neben sich
einige Cubes. Er wechselt zu seinem "`Pick'n'Place"'-Equipment und nimmt die Cubes auf. Dann baut er eine Br�cke aus den
Cubes, indem er sie in einer Linie vor sich platziert. Er �berquert die Br�cke und setzt seinen Weg fort.

\subsection{Hans und Peter im Zweikampf}
Hans und Peter spielen gemeinsam Cydonia. Peter ist in Team 1 und Hans in Team 2. Hans ist zur Flagge von Team 1
gelaufen und klaut die Flagge indem er hindurchl�uft. Peter, der gerade auf dem Weg zu Flagge von Team 2 war, bekommt
eine Warnung angezeigt, dass die Flagge seines Teams entwendet wurde. Er dreht um und l�uft Hans entgegen. Kurz nachdem
Hans sich auf den R�ckweg gemacht hat, trifft er auf Peter. Peter markiert mit seinem "`Swapper"'-Equipment zuerst Hans
und dann einen Cube etwas abseits des Weges. Hans befindet sich pl�tzlich an der Stelle, an der vorher der Cube gewesen
ist. Die Flagge, die er bei sich getragen hatte, ist an der urspr�nglichen Position geblieben, weshalb Peter sie nun
leicht aufnehmen und zur�ck zu seiner Basis bringen kann.

\subsection{Team 1 erzielt einen Punkt}
Peter spielt Cydonia. Er ist in Team 1 und hat soeben die Flagge von Team 2 geklaut. Er l�uft zur�ck zu seiner Basis. Da
die Flagge seines eigenen Teams gerade nicht in der Basis ist, kann er keinen Punkt machen. Er versteckt sich daher in
einer Ecke und baut mithilfe seines "`Pick'n'Place"'-Equipments eine kleine Wand, damit er nicht gesehen werden kann.
Max, ein Teamkollege von Peter, hat die Flagge von Team 1 zur�ckerobert bringt sie zur�ck zu seiner Basis. Peter bekommt
eine Meldung, dass die Flagge seines Teams wieder in der Basis ist. Er verl�sst sein Versteck und bringt die gegnerische
Flagge zu der Flagge seines Teams. Alle Spieler erhalten die Nachricht, dass Team 1 einen Punkt gemacht hat. Die Flagge
von Team 2 erscheint wieder an ihrem Ausgangspunkt in der Basis von Team 2.

\subsection{Peter und sein Team gewinnen}
Hans spielt Cydonia. Sein Team hat bereits vier Punkte f�r die Eroberung der gegnerischen Flagge erhalten. Er ist mit
der gegnerischen Flagge unterwegs zur�ck zu seiner Basis. Er bringt die gegnerische Flagge zur Flagge seines Teams und
erzielt damit den f�nften Punkt f�r sein Team. Das Spiel stopt und Peter sieht die Meldung, dass er und sein Team diese
Runde gewonnen haben. W�hrend er sich selbst lobend auf die Schulter klopft signalisiert ihm ein r�ckw�rts laufender
Counter, dass in wenigen Sekunden die n�chste Runde beginnt.
 