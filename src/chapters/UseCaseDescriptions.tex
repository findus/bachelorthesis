\chapter{Anwendungsf�lle}
\label{chapter:UseCaseDescriptions}

\section{Bewegen}
\begin{tabular}{|p{3cm}|p{9cm}|}
    \hline
	\textbf{Anwendungs-fallname} & Bewegen \\ \hline
	\textbf{Akteure} & Initialisiert von Spieler \\ \hline
	\multirow{7}{*}{\textbf{Ereignisfluss}} & 1) Der Spieler m�chte sich in der virtuellen Welt bewegen. Dazu visiert er sein Ziel an. \\[0.1cm]
	 & 2) Der Spieler kann vorw�rts oder r�ckw�rts gehen. \\[0.1cm]
	 & 3) Der Spieler kann seitw�rts gehen. \\[0.1cm]
	 & 4) Der Spieler kann springen. \\[0.1cm]
	 & \leftskip=1cm 5) Die Position des Spielers in der Welt �ndert sich entsprechend. \\ \hline
	\textbf{Anfangs-bedingungen} & keine Ahnung \\ \hline
	\textbf{Abschluss-bedingungen} & keine Ahnung \\ \hline
	\textbf{Qualit�ts-anforderungen} & keine Ahnung \\ \hline
\end{tabular}


\section{Equipment Wechseln}
\begin{tabular}{|p{3cm}|p{9cm}|}
    \hline
	\textbf{Anwendungs-fallname} & Epuipment wechseln \\ \hline
	\textbf{Akteure} & Initialisiert von Spieler \\ \hline
	\multirow{7}{*}{\textbf{Ereignisfluss}} & 1) Der Spieler m�chte ein anderes Equipment benutzen. Dazu wechselt er zu dem neuen Equipment. \\[0.1cm]
	 & \leftskip=1cm 2) Das bisherige Equipment wird deaktiviert, das neue Equipment aktiviert. der Spieler bekommt eine visuelle R�ckmeldung �ber den Wechsel. \\ \hline
	\textbf{Anfangs-bedingungen} & Der Spieler muss mehrere Equipments zur Verf�gung haben. \\ \hline
	\textbf{Abschluss-bedingungen} & Das neue Equipment muss aktiv und benutzbar sein. \\ \hline
	\textbf{Qualit�ts-anforderungen} & Alle Equipments m�ssen beim Wechsel ihren Zustand bewahren. \\ \hline
\end{tabular}


\newpage
\section{Cube bewegen}
\begin{tabular}{|p{3cm}|p{9cm}|}
    \hline
	\textbf{Anwendungs-fallname} & Cube bewegen \\ \hline
	\textbf{Akteure} & Initialisiert von Spieler \\ \hline
	\multirow{7}{*}{\textbf{Ereignisfluss}} & 1) Der Spieler m�chte einen Cube bewegen. Dazu visiert er den Cube an. \\[0.1cm]
	 & 2) Der Spieler benutzt die Funktion "`Aufnehmen"' des Equipments um den Cube aufzunehmen. \\[0.1cm]
	 & \leftskip=1cm 3) Der Cube wird aus der Spielwelt entfernt und erscheint im Inventar des Equipments. \\[0.1cm]
	 & 4) Der Spieler kann sich nun beliebig bewegen (siehe Anwendungsfall "`Bewegen"'). \\[0.1cm]
	 & 5) Der Spieler visiert die Stelle an, an der er den Cube platzieren m�chte. \\[0.1cm]
	 & 6) Der Spieler benutzt die Funktion "`Platzieren"' des Equipments um den Cube zu platzieren. \\[0.1cm]
	 & \leftskip=1cm 7) Der Cube verschwindet aus dem Inventar des Equipments und wird in die Spielwelt eingef�gt. \\ \hline
	\textbf{Anfangs-bedingungen} & Ein beweglicher Cube muss in Sichtweite des Spielers in der Spielwelt vorhanden sein.\newline Der Spieler muss das Pick'n'Place-Equipment ausgew�hlt haben. \newline Das Pick'n'Place-Equipment muss zur Aufnahme eines Cubes bereit sein. \\ \hline
	\textbf{Abschluss-bedingungen} & Der Cube muss in der Welt vorhanden sein. \newline Das Pick'n'Place-Equipment muss zur erneuten Aufnahme eines Cubes bereit sein. \\ \hline
	\textbf{Qualit�ts-anforderungen} & keine Ahnung \\ \hline
\end{tabular}

\newpage
\section{Gegnerische Flagge nehmen}
\begin{tabular}{|p{3cm}|p{9cm}|}
    \hline
	\textbf{Anwendungs-fallname} & Flagge nehmen \\ \hline
	\textbf{Akteure} & Initialisiert von Spieler \\ \hline
	\multirow{7}{*}{\textbf{Ereignisfluss}} & 1) Der Spieler m�chte die gegnerische Flagge aufnehmen. Dazu bewegt er sich zu der Flagge (siehe Anwendungsfall "`Bewegen"'). \\[0.1cm]
	 & \leftskip=1cm 2) Wenn der Spieler die Flagge ber�hrt, nimmt er sie auf. Der Spieler bekommt angezeigt, dass er die Flagge bei sich tr�gt. \\[0.1cm]
	 & 3) Der Spieler kann sich nun beliebig bewegen (siehe Anwendungsfall "`Bewegen"'). \\[0.1cm]
	 & \leftskip=1cm 4) Die Flagge bewegt sich mit dem Spieler mit. \\ \hline
	\textbf{Anfangs-bedingungen} & Kein Spieler darf die gegnerische Flagge bei sich tragen.\newline Die gegnerische Flagge muss f�r der Spieler erreichbar sein. \\ \hline
	\textbf{Abschluss-bedingungen} & Die Flagge muss sich mit dem Spieler bewegen. \\ \hline
	\textbf{Qualit�ts-anforderungen} & keine Ahnung \\ \hline
\end{tabular}

\newpage
\section{Gegnerische Flagge erobern}
\begin{tabular}{|p{3cm}|p{9cm}|}
    \hline
	\textbf{Anwendungs-fallname} & Einen Punkt erzielen \\ \hline
	\textbf{Akteure} & Initialisiert von Spieler \\ \hline
	\multirow{7}{*}{\textbf{Ereignisfluss}} & 1) Der Spieler m�chte die gegnerische Flagge erobern. Dazu bewegt sich zu seiner Flagge in seiner Basis (siehe Anwendungsfall "`Bewegen"'). \\[0.1cm]
	 & \leftskip=1cm 2) Ist die Flagge des Spielers in seiner Basis, erh�lt sein Team einen Punkt und die gegnerische Flagge wird zur�ckgesetzt. \\[0.1cm]
	 & 3) Ist die Flagge des Spielers nicht in seiner Basis, kann der Spieler warten, bis sie wieder da ist, und dann den Punkt erzielen. \\ \hline
	\textbf{Anfangs-bedingungen} & Der Spieler muss die gegnerische Flagge bei sich tragen (siehe Anwendungsfall "`Gegnerische Flagge nehmen"'). \\ \hline
	\textbf{Abschluss-bedingungen} & Die gegnerische Flagge muss sich wieder in ihrer Basis befinden. \\ \hline
	\textbf{Qualit�ts-anforderungen} & keine Ahnung \\ \hline
\end{tabular}


\newpage
\section{Objekte swappen}
\begin{tabular}{|p{3cm}|p{9cm}|}
    \hline
	\textbf{Anwendungs-fallname} & Objekte swappen \\ \hline
	\textbf{Akteure} & Initialisiert von Spieler \\ \hline
	\multirow{7}{*}{\textbf{Ereignisfluss}} & 1) Der Spieler m�chte zwei Objekte swappen. Dazu visiert er ein Ojbekt an. \\[0.1cm]
	 & 2) Der Spieler benutzt das Equipment das Objekt zu markieren. \\[0.1cm]
	 & \leftskip=1cm 3) Die Markierung wird visualisiert, indem das markierte Objekt f�r den Spieler hervorgehoben dargestellt wird. \\[0.1cm]
	 & 4) Der Spieler kann sich nun beliebig bewegen (siehe Anwendungsfall "`Bewegen"'). \\[0.1cm]
	 & 5) Der Spieler visiert ein weiteres Objekt an. \\[0.1cm]
	 & 6) Der Spieler benutzt das Equipment um das Objekt zu markieren. \\[0.1cm]
	 & \leftskip=1cm 7) Die beiden markierten Objekte tauschen ihre Positionen in der Spielwelt. \\[0.1cm]
	 & \leftskip=1cm 8) Wenn eines der Objekte ein gegnerischer Spieler ist und dieser die Flagge bei sich tr�gt, verliert er die Flagge. Sie bleibt an der urspr�nglichen Position des gegnerischen Spielers. \\ \hline
	\textbf{Anfangs-bedingungen} & Zwei Objekte, die der Spieler swappen kann, m�ssen sich in der Spielwelt befinden. \newline Der Spieler muss das Swapper-Equipment ausgew�hlt haben. \\ \hline
	\textbf{Abschluss-bedingungen} & Beide Objekte m�ssen an der jeweils anderen Position in der Spielwelt sein. \newline Die beiden Objekte m�ssen unmarkiert sein. \newline Die beiden Markierungen des Swapper-Equipments m�ssen wieder frei sein. \\ \hline
	\textbf{Qualit�ts-anforderungen} & Setzt der Spieler eine bereits gesetzte Markierung auf ein anderes Objekt, muss die Markierung des urspr�nglichen Objekts aufgehoben werden. \newline Ein Spieler darf die gegnerische Flagge nichtverlieren, wenn er von einem Spieler im \emph{gleichen} Team geswappet wird. \\ \hline
\end{tabular}


\newpage
\section{Eigene Flagge zur�ckholen}
\begin{tabular}{|p{3cm}|p{9cm}|}
    \hline
	\textbf{Anwendungs-fallname} & Eigene Flagge zur�ckholen \\ \hline
	\textbf{Akteure} & Initialisiert von Spieler \\ \hline
	\multirow{7}{*}{\textbf{Ereignisfluss}} & 1) Der Spieler m�chte die Flagge seines Teams zur�ckholen. Dazu bewegt er sich zu der Flagge (siehe Anwendungsfall "`Bewegen"'). \\[0.1cm]
	 & 2) Falls ein gegnerischer Spieler die Flagge bei sich tr�gt, muss der Spieler den gegnerischen Spieler swappen, damit dieser die Flagge verliert (siehe Anwendungsfall "`Objekte swappen"'). \\[0.1cm]
	 & \leftskip=1cm 3) Wenn der Spieler die Flagge ber�hrt, wird die Flagge zur�ckgesetzt. Der Spieler bekommt angezeigt, dass die Flagge wieder in der Basis ist. \\ \hline
	\textbf{Anfangs-bedingungen} & Die Flagge des Spielers darf sich nicht in ihrer Basis befinden. \\ \hline
	\textbf{Abschluss-bedingungen} & Die Flagge des Spielers muss sich in ihrer Basis befinden. \\ \hline
	\textbf{Qualit�ts-anforderungen} & keine Ahnung \\ \hline
\end{tabular}