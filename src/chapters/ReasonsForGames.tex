\section{Spielspa�}

Um zu verstehen, was ein gutes Computerspiel ausmacht, muss man untersuchen, warum sie �berhaupt genutzt werden. Dazu
soll in diesm Abschnitt ein kurzer �berblick �ber die wissenschaftliche Meinung gegeben werden.

\subsection{Motivation}
Zu den am h�ufigsten genannten Situationen, aus denen der Wunsch zum Computerspiel zu greifen, z�hlen Langeweile, Stress
(Wunsch nach Stressabbau), �rger oder Wut (Wunsch nach Aggressionsabbau), soziale Motive (zum Spielen
verabredet)\cite[S. 93]{Ladas_2002}\cite[S. 65ff]{Koehler_2008}. Die Spieler erwarten von der Nutzung
des Spiels also eine Ablenkung von ihrer realen Situation, etwas Spannendes zu erleben, und sich dadurch nachher besser
zu f�hlen als davor\cite[S. 66]{Koehler_2008}. Spieler w�hlen zu diesem Zweck vor allem Spiele, die eine
gewisse Verbindung zu ihrem Leben haben. Faktoren die dabei eine Rolle spielen sind\cite[S. 95]{Ladas_2002}:
\begin{itemize}
\item Assoziationen zu fr�heren Kinderspielen
\item Hobbys
\item Filmvorlieben
\item Wunsch nach Abenteuer, Erlebnissen
\item Ablehnung von Gewalt und Krieg
\item Ausleben von Ordnungssinn/�hnlichkeiten zur Organisation des 'realen' Lebens
\item Erinnerungen an bestimmte Lebenssituationen (belastend oder positiv)
\end{itemize}
Zur Motivation geh�rt aber auch der Wunsch nach Macht, Kontrolle, Leistung, Kompetenz und Erfolg: 
\begin{quotation}Aufgrund dieses Wunsches ist der Spieler immer wieder motiviert, sich mit dem Spiel zu besch�ftigen. Das Entwicklen erfolgreicher Schemata, das Unter-Kontrolle-Bringen von Situationen, das Erledigen von Aufgaben, Zeigen von Leistung, Erwerb von Kompetenz, um Kontrolle und Macht zu erlangen und erflogreich zu sein, sind sowohl Merkmale von Computerspielen als auch vom realen Leben.\cite[S. 74]{Koehler_2008}\end{quotation}

\subsection{Interaktivit�t}

