\section{Spielspa�}

Um zu verstehen, was ein gutes Computerspiel ausmacht, muss zuerst untersucht werden, warum sie �berhaupt genutzt
werden. Dazu wird im Folgenden ein kurzer �berblick �ber die derzeitige wissenschaftliche Meinung gegeben werden.

\subsection{Motivation}
Die am h�ufigsten genannten Situationen, aus denen der Wunsch, zum Computerspiel zu greifen, entsteht, sind Langeweile
(Wunsch nach Unterhaltung), Stress (Wunsch nach Stressabbau), �rger oder Wut (Wunsch nach Aggressionsabbau) und soziale
Motive (zum Spielen verabredet). \cite[S. 93]{Ladas_2002}\cite[S. 65ff]{Koehler_2008} Die Spieler erwarten von der
Nutzung des Spiels also eine Ablenkung von ihrer realen Situation, etwas Spannendes zu erleben, und sich dadurch nachher
besser zu f�hlen als davor. \cite[S. 66]{Koehler_2008}  Spieler w�hlen zu diesem Zweck vor allem Spiele, die eine
gewisse Verbindung zu ihrem Leben haben. Faktoren, die dabei eine Rolle spielen, sind laut \cite[S. 95]{Ladas_2002}:
\begin{itemize}
\item Assoziationen zu fr�heren Kinderspielen
\item Hobbys
\item Filmvorlieben
\item Wunsch nach Abenteuer, Erlebnissen
\item Ablehnung von Gewalt und Krieg
\item Ausleben von Ordnungssinn/�hnlichkeiten zur Organisation des 'realen' Lebens
\item Erinnerungen an bestimmte Lebenssituationen (belastend oder positiv)
\end{itemize}
Zur Motivation geh�rt aber auch der Wunsch nach Macht, Kontrolle, Leistung, Kompetenz und Erfolg: 
\begin{quotation}
Aufgrund dieses Wunsches ist der Spieler immer wieder motiviert, sich mit dem Spiel zu besch�ftigen. Das Entwickeln erfolgreicher Schemata, das Unter-Kontrolle-Bringen von Situationen, das Erledigen von Aufgaben, Zeigen von Leistung, Erwerb von Kompetenz, um Kontrolle und Macht zu erlangen und erfolgreich zu sein, sind sowohl Merkmale von Computerspielen als auch vom realen Leben. \cite[S. 74]{Koehler_2008}
\end{quotation}

\subsection{Interaktivit�t}
Computerspiele unterscheiden sich von anderen audiovisuellen Medien vor allem durch eine unabdingbare Notwendigkeit der
Steuerung durch den Benutzer. \cite[59]{Ladas_2002} Diese Interaktivit�t sehen die meisten Forscher als zentrale
Komponente des Unterhaltungserlebens der Spieler an. \cite{Klimmt_2006} Wesentlich dabei ist die direkte Reaktion auf
jede Eingabe des Spielers. Die Wichtigkeit direkten Feedbacks auf Benutzereingaben betont auch die Firma Apple, die von
vielen als Vorreiter in Sachen Benutzerschnittstellen bezeichnet wird, in ihrer "`iOS human interface guidelines"'.
\cite[20]{Apple_2012} Die direkte Reaktion auf Eingaben, in Spielen meist in Form einer Ver�nderung innerhalb der
virtuellen Umgebung, gibt dem Spieler das Gef�hl, etwas zu bewirken. Die Motivationspsychologie stellt den Begriff
"`effectance"' \cite{White_1959}, der den positiven emotionalen Einfluss von Selbswirksamkeitserfahrungen beschreibt.
Computerspiele verursachen starke Effectance-Erfahrungen, indem sie den Spieler meist die Spielwelt mit wenigen Eingaben
wesentlich ver�ndern lassen. \cite{Klimmt_2006_Effectance} Da Computerspiele stets verl�sslich direkte Reaktionen auf
Eingaben des Spielers geben, rufen sie bei Spielern kontinuierlich Effectance-Erfahrungen hervor. \cite{Klimmt_2006}
Selbstwirksamkeitserfahrungen erleben die Spieler folglich des interaktiven Charakters von Computerspielen, weshalb
dieser als wesentlicher Aspekt des Spielspa�es betrachtet werden muss.

\subsection{Leistungshandeln}
Spiele verlangen vom Spieler meist, ein schwieriges Problem zu l�sen oder Aufgaben zu bew�ltigen. \cite{Oerter_1999} Bei
Multiplayer-Computerspielen kann beides gleichzeitig gefordert sein, wenn als Aufgabe das Erf�llen des Spielziels und
als Problem das dem eigenen entgegenstehende Spielziel des Gegners - und somit letztlich der Gegner an sich - gesehen
wird. Der Spieler muss also etwas leisten:
\begin{quotation}
Computerspiele fordern (dauerhaft) gute Leistungen und sie belohnen Leistungen mit verschiedenen Varianten von Belohnungen,[\ldots]. Sie bestrafen unzureichende Leistungen aber auch mit frustrierenden Erfahrungen. \cite{Behr_2008}
\end{quotation}
Die eintretenden Erfolge und Misserfolge rufen spezifische emotionale Konsequenzen hervor.
Erfolge, die auf die eigene F�higkeit zur�ckgef�hrt werden, erzeugen zum Beispiel Zuversicht und das Gef�hl der
Kompetenz. \cite{Behr_2008} Der Spieler wird also stets versucht sein, Erfolge zu maximieren und das Eintreten von Misserfolgen zu verhindern.
