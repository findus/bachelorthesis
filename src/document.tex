\documentclass{article}

\usepackage[utf8]{inputenc}

\begin{document}


\section{Cydonia 43: Ein gewaltfreier "Multiplayer-Taktik-Shooter"}

\subsection{Motivation}

Seit langem herrscht in Politik und Gesellschaft Uneinigkeit über die Auswirkungen von Gewaltdarstellungen in
Computerspielen. Dennoch sind solche stets in vielen Spielen enthalten gewesen.

\subsection{Wahl der Programmiersprache}

Im Bereich der Computerspiele wird heute überwiegend C++ eingesetzt. Hauptgrund dafür ist die hohe Performance, außerdem
die OO-Unterstützung und die Tatsache, dass die meisten Tools und Bibliotheken auf C++ basieren oder dazu kompatibel
sind. Da das Ziel dieser Arbeit allerdings nicht die Entwicklung eines marktreifen AAA-Games, sondern vielmehr eines
Prototypen war, konnte der Performance-Faktor weitgehend vernachlässigt werden. Bereits vorhandene Erfahrung im Umgang
mit der Programmiersprache, den verfügbaren Libraries und der Entwicklungsumgebung, sowie die hohe Produktivität der
Sprache sind entscheidend für die Wahl von Java verantwortlich.

\subsection{Wahl der Engine}

\subsubsection{Was ist eine Game-Engine?}

Ein Computerspiel besteht meist aus einer Spielewelt, die optisch und aktustisch erlebt werden kann. In vielen Spielen
steuert der Spieler seinen Hauptcharakter, den sog. Avatar, durch diese Welt. Dazu werden Softwarebibliotheken benötigt,
die dreidimensionalen Raumklang erzeugen, Objekte grafisch darstellen, physikalische Abläufe berechnen, Benutzereingaben
in Steuerbefehle umwandeln, sowie Ereignisse über das Netzwerk kommunizieren. Eine Spiele-Engine bietet dem
Entwickler diese und andere Funktionalitäten über ein einheitliches Interface an. Zum einen wird dadurch der
Programmcode besser strukturiert, zum anderen der Entwickler entlastet.
Es bietet sich dabei natürlich an, eine Engine für mehrere (ähnliche) Spiele zu verwenden. Einige Hersteller stellen
ihre Engine anderen Entwicklern (teils gegen Lizenzgebühren) zur Verfügung. Bekannte Engines großer Spielehersteller
sind die Frostbite-Engine (Dice), Cry-Engine (Crytec), Unreal-Engine (Epic), die neben dem Hauptspiel, für das sie
entwickelt wurden, auch in weiteren Spielen verwendet wurden.

Es ist natürlich möglich, ein Spiel ohne bereits existierende Game-Engine zu schreiben. Das führt allerdings in den
meisten Fällen dazu, dass man eine eigene Engine innerhalb des Spiels entwickelt. Da heutzutage reichlich gute Engines
verfügbar sind, einige davon sogar ohne Lizenzgebühren frei verwendbar, macht dieser Mehraufwand wenig Sinn. Vor allem
ist zur Entwicklung einer guten Engine viel Erfahrung nötig.

\subsubsection{Game-Engine für Java}

Die meisten Game-Engines sind in C/C++ geschrieben. Eine solche Engine als Teil einer Java Software zu verwenden wäre
über das JNI evtl. möglich, ist aber sicher keine einfache Lösung. Zum Glück gibt es auch einige Engines die zu Java
kompatibel oder gar in Java geschrieben sind.
Die Auswahl ist nicht sehr groß, so lässt sich zum etwa in der Auflistung der Wikipedia schnell die Java MonkeyEngine
bestes Gesamtpaket identifizieren.

\end{document}
